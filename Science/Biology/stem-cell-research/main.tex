% reference manager: https://zbib.org/337c8b21d29f4e798bb79084d2091dbf

\documentclass[a4paper]{article}
\usepackage[utf8]{inputenc}
\usepackage{url}



\title{Stem cell research}
\author{Thomas Holland}
\date{November 2020}

\begin{document}

\maketitle
\section*{Questions}
\begin{enumerate}
    \item \textbf{Describe the unique properties of stem cells? \cite{NationalInstitutesofHealth}}
    
\begin{itemize}
    \item Have the ability to self renew

This is unlike normal cells which normally are unable to replicate. The resultant two cells may be stem cells or differentiate into specialised cells. What causes the differentiation is yet unknown. 

    \item Stem cells have the ability to recreate functional tissues
    
Different types of stem cells have differing ability to differentiate into specialised cells. This is caused by the fact that they may have defining morphological features and patterns of gene expression reflective of that tissue.
\end{itemize}

    \item\textbf{ Describe the principle arguments behind the use of stem cells? \cite{YvetteBrazier2018}}

First, with the right stimulation, many stem cells can take on the role of any type of cell, and they can regenerate damaged tissue, under the right conditions.

This potential could save lives or repair wounds and tissue damage in people after an illness or injury. Scientists see many possible uses for stem cells.

Several uses include:
\begin{itemize}
    \item Tissue regeneration
    \item Cardiovascular disease treatment
    \item Brain disease treatment
    \item Blood disease treatments
\end{itemize}
    \item \textbf{What are multipotent stem cells? Give examples. \cite{IanMurnaghan}} 
    
A multipotent stem cell can give rise to other types of cells but it is limited in its ability to differentiate. These other types of cells are also limited in numbers. This means that multipotent stem cells are essentially committed to produce specific cell types.

Multipotent stem cells are found in the tissues of adult mammals. It is thought that they are in most body organs, where they replace diseased or aged cells. Thus, they function to replenish the body's cells throughout an individual's life.
    \item \textbf{What are the key benefits of multipotent stem cells? \cite{IanMurnaghan}}
    
\begin{itemize}
    \item These stem cells have already partially differentiated and they continue specialising as they develop.
    \item In transplants they can be isolated, albeit often with difficulty, from a person's tissues and then guided to develop into a certain type of cell, before being transplanted back into the same patient.
    \item Avoids the immunological challenges of pluripotent foetal stem cell usage, where a patient's immune system could potentially reject a 'foreign' tissue. 
    \item The ethical debate and controversy involved in extracting foetal stem cells is avoided, because neither foetal tissues nor an aborted embryo are necessary for treatments.
\end{itemize}

    \item \textbf{What are totipotent stem cells? \cite{Cheng2008}}

Totipotent stem cells have the capacity to produce all adult cell types, can enter the germ line (i.e. contribute genetic material to succeeding generations), and have proven ability to self-replicate (i.e. produce daughter cells that are identical to the parent).


    \item \textbf{How are they [totipotent cells] different to multipotent stem cells? \cite{Cheng2008}}

Unlike the multipotent stem cells, the totipotent cells are able to produce all adult cell types. 

    \item \textbf{What special qualities do totipotent stem cells have? \cite{Cheng2008}}

\begin{itemize}
	\item Can enter the germ line: Contribute genetic materials to succeeding generations
	\item Have a proven ability to self replicate
	\item Only found in early embryonic tissues (so typically from IVF)
\end{itemize}


    \item \textbf{What are the key benefits of totipotent stem cell use?}


    \item \textbf{What are pluripotent stem cells?}


    \item \textbf{How are they different to totipotent stem cells?}


    \item \textbf{What are the key benefits of pluripotent stem cell use?}


    \item \textbf{What is Stem Cell Therapy?}


    \item \textbf{What are the main benefits of Stem Cell Therapy?}


    \item \textbf{How can stem cells be used to treat Multiple Sclerosis?}


    \item \textbf{Elaborate on the benefits of Stem Cells.}


    \item \textbf{Identify and explain the use of Stem Cells in treating 4 different problems.}


    \item \textbf{What are the key concerns regarding the use of Stem Cells? Explain why they are a concern.}


    \item \textbf{Explain the main areas of concern regarding the safety of Stem Cell use.}


    \item \textbf{Explain how induced pluripotent stem (IPS) cells are produced and the advantages of this method.}


\end{enumerate}

\medskip

\bibliographystyle{plainurl}
\bibliography{../../../bib/Science-Biology-stem_cell_research}
\end{document}
