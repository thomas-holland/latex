\documentclass[a4paper]{article}
\usepackage[utf8]{inputenc}
\usepackage{url}



\title{Stem cell research}
\author{Thomas Holland}
\date{November 2020}

\begin{document}


\maketitle
\section*{Questions}
\begin{enumerate}
    \item \textbf{Describe the unique properties of stem cells? \cite{NationalInstitutesofHealth}}
    
\begin{itemize}
    \item Have the ability to self renew

This is unlike normal cells which normally are unable to replicate. The resultant two cells may be stem cells or differentiate into specialised cells. What causes the differentiation is yet unknown. 

    \item Stem cells have the ability to recreate functional tissues
    
Different types of stem cells have differing ability to differentiate into specialised cells. This is caused by the fact that they may have defining morphological features and patterns of gene expression reflective of that tissue.
\end{itemize}

    \item\textbf{ Describe the principle arguments behind the use of stem cells? \cite{YvetteBrazier2018}}

First, with the right stimulation, many stem cells can take on the role of any type of cell, and they can regenerate damaged tissue, under the right conditions.

This potential could save lives or repair wounds and tissue damage in people after an illness or injury. Scientists see many possible uses for stem cells.

Several uses include:
\begin{itemize}
    \item Tissue regeneration
    \item Cardiovascular disease treatment
    \item Brain disease treatment
    \item Blood disease treatments
\end{itemize}
    \item \textbf{What are multipotent stem cells? Give examples. \cite{IanMurnaghan}} 
    
A multipotent stem cell can give rise to other types of cells but it is limited in its ability to differentiate. These other types of cells are also limited in numbers. This means that multipotent stem cells are essentially committed to produce specific cell types.

Multipotent stem cells are found in the tissues of adult mammals. It is thought that they are in most body organs, where they replace diseased or aged cells. Thus, they function to replenish the body's cells throughout an individual's life.
    \item \textbf{What are the key benefits of multipotent stem cells? \cite{IanMurnaghan}}
    
\begin{itemize}
    \item These stem cells have already partially differentiated and they continue specialising as they develop.
    \item In transplants they can be isolated, albeit often with difficulty, from a person's tissues and then guided to develop into a certain type of cell, before being transplanted back into the same patient.
    \item Avoids the immunological challenges of pluripotent foetal stem cell usage, where a patient's immune system could potentially reject a 'foreign' tissue. 
    \item The ethical debate and controversy involved in extracting foetal stem cells is avoided, because neither foetal tissues nor an aborted embryo are necessary for treatments.
\end{itemize}

    \item \textbf{What are totipotent stem cells? \cite{Cheng2008}}

Totipotent stem cells have the capacity to produce all adult cell types, can enter the germ line (i.e. contribute genetic material to succeeding generations), and have proven ability to self-replicate (i.e. produce daughter cells that are identical to the parent).


    \item \textbf{How are they [totipotent cells] different to multipotent stem cells? \cite{Cheng2008}}

Unlike the multipotent stem cells, the totipotent cells are able to produce all adult cell types. 

    \item \textbf{What special qualities do totipotent stem cells have? \cite{Cheng2008}}

\begin{itemize}
	\item Can enter the germ line: Contribute genetic materials to succeeding generations
	\item Have a proven ability to self replicate
	\item Only found in early embryonic tissues (so typically from IVF)
\end{itemize}


    \item \textbf{What are pluripotent stem cells? \cite{William}}
    
    Cells that are self-replicating, are derived from human embryos or human foetal tissue, and are known to develop into cells and tissues of the three primary germ layers. Although human pluripotent stem cells may be derived from embryos or foetal tissue, such stem cells are not themselves embryos.


    \item \textbf{How are they different to totipotent stem cells? \cite{DifferenceBetween}}

Totipotent cells have the ability to form any cell type at any stage of development, whereas pluripotent cells have the ability to form any cell type after first few cleavages of embryo.

    \item \textbf{What are the key benefits of pluripotent stem cell use? \cite{ukessays.com2018}}

\begin{itemize}
	\item Pluripotent stem cells give a renewable basis of healthy cells \& tissues to treat many type of diseases similar to heart disease and diabetes.
	\item People who are burn \& those patients who suffer from autoimmune diseases like Parkinson’s can give advantage from the usage of pluripotent stem cells.
	\item Pluripotent stem cells have large potential for treatment of diseases, because they give rise to majority of cell types in human body, which include muscle, blood, heart \& nerve cells.
	\item The use for pluripotent stem cells include the generation of cells \& tissues that are use in transplantation.
	\item Drug study \& research next method that pluripotent stem cells are beneficial. Animals are mostly used to measure the safety and use of drugs. Those drugs which are secure and used in development for testing on animals.
\end{itemize}

    \item \textbf{What is Stem Cell Therapy? \cite{Mahla2016}}
   
Stem-cell therapy is the use of stem cells to treat or prevent a disease or condition. It is normally completed with a bone marrow transplant with cells from the umbilical cord blood. 

    \item \textbf{What are the main benefits of Stem Cell Therapy? \cite{Drparkercom2017} }

\begin{itemize}
	\item Avoid surgery and its many complications and risks: Stem cell therapy is a minimally invasive, non-surgical procedure. The stem cells are harvested from the patient’s bone marrow from the iliac crest (pelvis).
	\item Minimal post-procedural recovery time: One of the most time consuming factors of any injury is not always the treatment itself, but actually the recovery time. With stem cell therapy, recovery time is minimal.
	\item No use of general anesthesia: Do you not like the way general anesthesia makes your feel? Or do you simply get anxious at the thought of being put under? Stem cell therapy may be just what you need as it does not require the use of general anesthesia.
	\item No risk of rejection: Due to using biologics extracted from the patient, there is no risk of rejection.
	\item No communicable disease transmission: As the cells originate within your own body, there is no risk of spreading disease from or to another person.
\end{itemize}

    \item \textbf{Identify and explain the use of Stem Cells in treating 4 different problems.}
\begin{itemize}
\item[Multiple Sclerosis:] Usage of bone marrow transplant, after a session of chemotherapy \cite{WebMD2019}.
\item[Parkinsons:] A typical treatment at one of these clinics involves removing fat cells from the abdomen (some clinics remove bone marrow or blood for this procedure), treating the cells in various ways in order to isolate mesenchymal stem cells or stromal cells from the removed tissue, and finally injecting these cells back into the body. The cells are re-introduced into the body in different locations (into the bloodstream, cerebral spinal fluid, nose, eye, etc.) depending on which disease is being targeted \cite{REBECCAGILBERT2018}.
\item[Brain Cancer:] One method includes taking immature nerve cells from the bone marrow and introducing them to the brain. Neural stem cells can recognize signals from tumor cells in the brain and track cancer cells as they move around the brain. \cite{HealthGuideInfo2010}
\item[Testicular Cancer:] Stem cell transplant allows doctors to use higher doses of chemo. Stem cells used
to be taken from the bone marrow, but this is done less often now. In the weeks before
treatment, a special machine collects blood-forming stem cells from the patient's
bloodstream. \cite{Cancer.org2018}
\end{itemize}


    \item \textbf{What are the key concerns regarding the use of Stem Cells? Explain why they are a concern. \cite{Rickard2002}}

The main concerns with using stem cells are ethical considerations. There is a case to be made that the harvesting of human embryonic stem cells violates the value of life in that it results in the destruction of human life with value (i.e. human embryos).


    \item \textbf{Explain the main areas of concern regarding the safety of Stem Cell use. \cite{Volarevic2018}}

The pluripotency of stem cells is a double-edged sword; the same plasticity that permits hESCs to generate hundreds of different cell types also makes them difficult to control after in vivo transplantation.

    \item \textbf{Explain how induced pluripotent stem (IPS) cells are produced and the advantages of this method. \cite{Surat2019}}

Induced pluripotent stem cells (iPS) are somatic cells that can be reprogrammed by expressing a combination of embryonic transcription factors. Like embryonic stem cells, iPS cells can differentiate into all germ cell layers. The reprogrammed cells can be used to generate stem cells for diseases, drug development, and personalized regenerative stem cell therapy. 

Induced pluripotent stem cells are not derived from embryonic stem cells, and this negates the ethical concerns present in the field regarding the utilization of embryonic stem cells, however, the main issue is the use of retroviruses to generate iPSCs as they are associated with cancer.


\end{enumerate}

\medskip

\bibliographystyle{plainurl}
\bibliography{../../../bib/Science-Biology-stem_cell_research}
\end{document}
